\section{EXTERNAL RELATIONS POLICIES OF INTEREST TO THE FACULTY}
	\subsection{Communications and Publications}

		The Office of Marketing and Communications provides assistance in
		developing brochures, distributing press releases, advertising events,
		and ordering stationery and business cards. Their staff members also
		produce the Westmont magazine, the fall and spring events calendars, and
		serve as a liaison between journalists participating in ProfNet searches
		and Westmont faculty.  The Office of Marketing and Communications also
		curates a collection of official Westmont logos, fonts, color schemes,
		electronic letterhead and style guides for use by faculty and staff
		creating branded materials.  For more information, visit
		\href{https://www.westmont.edu/comm}{ \url{https://www.westmont.edu/comm}}.
	\subsection{Grants Policy}
		\begin{enumerate}[label=\alph*)]
			\item{\underline{Grants from Private
					Foundations}:  All grant applications made to private foundations must be processed through Westmont College.  This requirement, and the steps outlined below, is necessitated by the need by the College to determine priorities as it makes requests and to ensure oversight of approved faculty projects by the Provost, in consultation with the Office of College Advancement.
				\begin{enumerate}[label=\alph*)]
					\item{Approval to make a grant application
						must be secured first from
						the office of the Provost by
						completing the Intent to
						Apply for Extramural Funding
						form accessed online through
						the Provost's webpage under
						Faculty Development.}
					\item{The faculty member will, in most
						cases, draft and submit the
						grant proposal.}
				\end{enumerate}
			}
			\item{\underline{Grants from Governmental Agencies}:  Grant applications seeking funding from governmental and quasi-governmental agencies (e.g., the National Science Foundation) are governed by regulations and policies specific to the funding sources. These applications are processed through the office of the Provost.
			}
		\end{enumerate}
	\subsection{Patent/Copyright Policy}

		Unauthorized use of trademarked names or symbols, including Westmont's, is
		prohibited.  Where College resources are used, the College retains ownership
		of all faculty, staff and student inventions and other intellectual property
		that may be patented, copyrighted, trademarked or licensed for commercial
		purposes.

		Faculty use of copyrighted material is governed by applicable U.S.
		copyright law.  Federal copyright law applies to all forms of
		information, including electronic communications.  Violations of
		copyright laws include, but are not limited to, making unauthorized
		copies of any copyrighted material (including software, text, images,
		audio, and video), and displaying or distributing copyrighted materials
		over computer networks without the author's permission except as
		provided in limited form by copyright fair use restrictions.  The ``fair
		use'' provision of the copyright law allows for limited reproduction and
		distribution of published works without permission for such purposes as
		criticism, news reporting, teaching (including multiple copies for
		classroom use), scholarship, or research.  Copyright guidelines,
		available from the College Bookstore, are outlined in ``Questions and
		Answers on Copyright for the Campus Community'' (National Association of
		College Stores and the Association of American Publishers, 1997).

	\subsection{Use of Institutional Letterhead, Trademarks, Tradenames}
		The logo, letterhead, and other insignia of the College are approved by the President.  College insignia or other identifying symbols should be used only for official business on behalf of the College.  Private use of College symbols and stationery that might imply institutional endorsement of a faculty member's activities must first receive approval of the Provost.
	\subsection{Political Activity}
		As an academic institution Westmont College maintains a non-partisan position and ensures the freedom of faculty as citizens to engage in political speech and activities so far as they are able to do so consistent with their obligations as teachers and scholars.  However, when speaking or acting as private citizens, faculty must avoid creating the impression they are doing so for the College.
	\subsection{Fundraising}
		College fundraising activities are conducted through the Office of College Advancement.  Individual fundraising efforts by faculty may be conducted only with the express approval of the Vice President for Advancement and Information Technology.
