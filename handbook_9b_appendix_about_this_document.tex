\section{About this document}
	\label{sec:AboutThisDocument}

	This document is an official governing document for Westmont College so changes are carefully managed and subject to a formal approval process
	(See Section
	\ref{sec:ProtocolsForRevision},
	``\nameref{sec:ProtocolsForRevision}''
	) .
	To support the integrity of the document, it is stored in a Github repository which functions as its change management system and supports transparency into any document modifications.
	The original LaTeX source documents can be found in that repository located
	\href{https://github.com/jaron-burdick/WestmontFacultyHandbook}{here}.


	\subsection{Current Version}


		To make sure that you have the latest version of this document, compare the version and date in the
		footer of this document with the dates in the footers of the current versions that are online.
		All released versions can be found in
		\href{https://github.com/jaron-burdick/WestmontFacultyHandbook/tree/Official-2023-2024-RC/releases}{the releases} folder of the Github repository.
		including:
		\begin{itemize}
			\item{The Official 2023-2024 version: \href{https://github.com/djp3/WestmontFacultyHandbook/tree/Official-2023-2024-RC/releases}{handbook.2023-2024.pdf}}
			\item{Historical versions for reference, labelled as such}
		\end{itemize}

	\subsection{Proposing Changes}
		\subsubsection{Technical Mechanism for Changes}
			Using the features of github, it is possible to submit changes
			directly to the source text of this document by editting
			it online. To accomplish this,
			navigate to
			\href{https://github.com/jaron-burdick/WestmontFacultyHandbook}{the canonical github repository}.
			and
			edit the files that end in ``.tex''

			Such edits will be noted as proposed changes and are called ``pull requests'' by github.
			The changes will need to be voted on by the Faculty Council before the pull request will be accepted, thus changing the document.

			Alternatively, one can open an ``issue'' on the website to propose a change in general terms.  The issue has its own
			comment thread.

			Once the changes are approved via a formal process, the document must be typeset and submitted to the repository for distribution.
		\subsubsection{Techno-social Mechanism for Changes Initiated by Faculty}
			\begin{center}
				\includegraphics[width=\textwidth]{images/change_process.png}
			\end{center}





